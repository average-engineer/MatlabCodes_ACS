\section{System perturbation}
\begin{figure}[h]
  \centering
  \tikzstyle{block}     = [draw, rectangle, minimum height=1cm, minimum width=1.6cm]
    \tikzstyle{branch}    = [circle, inner sep=0pt, minimum size=1mm, fill=black, draw=black]
    \tikzstyle{connector} = [->, thick]
    \tikzstyle{dummy}     = [inner sep=0pt, minimum size=0pt]
    \tikzstyle{inout}     = []
    \tikzstyle{sum}       = [circle, inner sep=0pt, minimum size=2mm, draw=black, thick]
    \begin{tikzpicture}[auto, node distance=1.6cm, >=stealth']
      \node[block] (G) {$\mathbf{G}$};
      \node[block] (K) [below=of G] {$\mathbf{K}$};
      \node[sum] (s1) [left=of G] {};
      \node[inout] (r) [left=of s1] {$\mathbf{w_1}$};
      \node[sum] (s3) [right= of K] {};
      \node[inout] (n) [right=of s3] {$\mathbf{w_2}$};
      \draw[connector] (s1) -- node{$\mathbf{e}_1$}(G);
      \draw[connector] (G) -| node{} (s3);
      \draw[connector] (n) -- (s3);
      \draw[connector] (r) -- (s1);
      \draw[connector] (s3) -- node{$\mathbf{e}_2$} (K);
      \draw[connector] (K) -| node{} (s1);
    \end{tikzpicture}
	  \caption{Feedback Control Loop.}
    \label{fig:blockDiagram1}
\end{figure}

The closed-loop transfer functions can be formulated as:
\begin{align*}
    \begin{bmatrix}
    \bm{e}_1 \\
    \bm{e}_2
    \end{bmatrix} =
    \begin{bmatrix}
    \bm{H}_1 & \bm{H}_2\\
    \bm{H}_3 & \bm{H}_4
    \end{bmatrix}
    \begin{bmatrix}
    \bm{w}_1 \\
    \bm{w}_2
    \end{bmatrix}
\end{align*}
For the feedback system, 
\begin{align*}
    \bm{e}_1 
    &= \bm{w}_1 + \bm{K}\bm{e}_2\\
    &= \bm{w}_1 + \bm{K}\left(\bm{w}_2 + \bm{G}\bm{e}_1\right)\\
    &= \left(\bm{I} - \bm{KG}\right)^{-1}\bm{w}_1 + \bm{K}\left(\bm{I} - \bm{GK}\right)^{-1}\bm{w}_2
    \Leftrightarrow \bm{H}_1 = \left(\bm{I} - \bm{KG}\right)^{-1}, 
    \bm{H}_2 = \bm{K}\left(\bm{I} - \bm{GK}\right)^{-1}
\end{align*}
where
\begin{align*}
    \bm{G} = 
    \begin{bmatrix}
        7&8\\
        6&7
    \end{bmatrix}
    \begin{bmatrix}
        \frac{1}{s+1}&0\\
        0&\frac{2}{s+2}
    \end{bmatrix}
    \begin{bmatrix}
        7&8\\
        6&7
    \end{bmatrix}^{-1}
\end{align*}
Similarly, we can derive $\bm{H}_3$ and $\bm{H}_4$ and our closed loop system becomes 
\begin{align*}
    \begin{bmatrix}
    \bm{e}_1 \\
    \bm{e}_2
    \end{bmatrix} =
    \begin{bmatrix}
    \left(\bm{I} - \bm{KG}\right)^{-1} & \bm{K}\left(\bm{I} - \bm{GK}\right)^{-1}\\
    \bm{G}\left(\bm{I} - \bm{KG}\right)^{-1} & \left(\bm{I} - \bm{GK}\right)^{-1}
    \end{bmatrix}
    \begin{bmatrix}
    \bm{w}_1 \\
    \bm{w}_2
    \end{bmatrix}
\end{align*}
\begin{figure}[h!]
    \centering
    \scalebox{0.6}{
    \begin{tikzpicture}
        % This file was created by matlab2tikz.
%
%The latest updates can be retrieved from
%  http://www.mathworks.com/matlabcentral/fileexchange/22022-matlab2tikz-matlab2tikz
%where you can also make suggestions and rate matlab2tikz.
%
\begin{tikzpicture}

\begin{axis}[%
width=4.521in,
height=3.566in,
at={(0.758in,0.481in)},
scale only axis,
xmin=0,
xmax=10,
xlabel style={font=\color{white!15!black}},
xlabel={Time(s)},
ymin=-4,
ymax=12,
ylabel style={font=\color{white!15!black}},
ylabel={Amplitude},
axis background/.style={fill=white},
xmajorgrids,
ymajorgrids,
legend style={legend cell align=left, align=left, draw=white!15!black}
]
\addplot [color=red, line width=2.0pt]
  table[row sep=crcr]{%
0	2\\
0.101010101010101	8.42623970826981\\
0.202020202020202	10.7621057231896\\
0.303030303030303	10.7878215819141\\
0.404040404040404	9.60888112751283\\
0.505050505050505	7.89608505093927\\
0.606060606060606	6.04285309680178\\
0.707070707070707	4.26785130992713\\
0.808080808080808	2.68155400683928\\
0.909090909090909	1.32908972854259\\
1.01010101010101	0.217549150638849\\
1.11111111111111	-0.66683966845458\\
1.21212121212121	-1.34810436928188\\
1.31313131313131	-1.85414633341325\\
1.41414141414141	-2.21304879244675\\
1.51515151515152	-2.45107442528099\\
1.61616161616162	-2.59170414197025\\
1.71717171717172	-2.65530031260555\\
1.81818181818182	-2.65912933256417\\
1.91919191919192	-2.6175772310613\\
2.02020202020202	-2.54245602767167\\
2.12121212121212	-2.44333965674554\\
2.22222222222222	-2.32789441422257\\
2.32323232323232	-2.20218525633379\\
2.42424242424242	-2.07094933641642\\
2.52525252525253	-1.93783416582009\\
2.62626262626263	-1.80560120773626\\
2.72727272727273	-1.67629752269453\\
2.82828282828283	-1.55139890526133\\
2.92929292929293	-1.43192818350378\\
3.03030303030303	-1.31855225016577\\
3.13131313131313	-1.21166111615268\\
3.23232323232323	-1.1114319201484\\
3.33333333333333	-1.01788045079628\\
3.43434343434343	-0.930902372603511\\
3.53535353535354	-0.850306010658092\\
3.63636363636364	-0.775838249885523\\
3.73737373737374	-0.707204843746969\\
3.83838383838384	-0.644086203617851\\
3.93939393939394	-0.586149550548348\\
4.04040404040404	-0.533058151939439\\
4.14141414141414	-0.484478232952611\\
4.24242424242424	-0.440084042442841\\
4.34343434343434	-0.399561462411537\\
4.44444444444444	-0.362610475350303\\
4.54545454545455	-0.328946742715921\\
4.64646464646465	-0.298302497850615\\
4.74747474747475	-0.270426915996328\\
4.84848484848485	-0.245086091013386\\
4.94949494949495	-0.222062721637693\\
5.05050505050505	-0.201155588462204\\
5.15151515151515	-0.1821788853694\\
5.25252525252525	-0.164961455097774\\
5.35353535353535	-0.149345967360594\\
5.45454545454545	-0.135188068927219\\
5.55555555555556	-0.122355527898746\\
5.65656565656566	-0.1107273887114\\
5.75757575757576	-0.100193149898029\\
5.85858585858586	-0.0906519730982378\\
5.95959595959596	-0.0820119290423\\
6.06060606060606	-0.0741892840898295\\
6.16161616161616	-0.067107829256822\\
6.26262626262626	-0.060698252413514\\
6.36363636363636	-0.0548975533990715\\
6.46464646464646	-0.0496485011117411\\
6.56565656565657	-0.0448991311417991\\
6.66666666666667	-0.0406022821768567\\
6.76767676767677	-0.0367151691906455\\
6.86868686868687	-0.0331989912999926\\
6.96969696969697	-0.0300185721184663\\
7.07070707070707	-0.0271420304317907\\
7.17171717171717	-0.0245404790558387\\
7.27272727272727	-0.0221877498019722\\
7.37373737373737	-0.0200601425582032\\
7.47474747474747	-0.01813619659147\\
7.57575757575758	-0.0163964822811225\\
7.67676767676768	-0.014823411602574\\
7.77777777777778	-0.0134010657900189\\
7.87878787878788	-0.0121150387159275\\
7.97979797979798	-0.0109522946310953\\
8.08080808080808	-0.00990103901119349\\
8.18181818181818	-0.00895060135326855\\
8.28282828282828	-0.00809132885797528\\
8.38383838383838	-0.00731449002022933\\
8.48484848484848	-0.00661218723232241\\
8.58585858585859	-0.00597727757937968\\
8.68686868686869	-0.00540330107746183\\
8.78787878787879	-0.00488441566980285\\
8.88888888888889	-0.00441533835685085\\
8.98989898989899	-0.0039912918911968\\
9.09090909090909	-0.00360795651940397\\
9.19191919191919	-0.00326142629946873\\
9.29292929292929	-0.00294816956543012\\
9.39393939393939	-0.00266499314977743\\
9.49494949494949	-0.00240901001004923\\
9.5959595959596	-0.00217760993863173\\
9.6969696969697	-0.00196843306449397\\
9.7979797979798	-0.0017793458826733\\
9.8989898989899	-0.00160841957196559\\
10	-0.00145391038368362\\
};
\addlegendentry{y1}

\addplot [color=black, line width=2.0pt]
  table[row sep=crcr]{%
0	0\\
0.101010101010101	5.9432017948394\\
0.202020202020202	8.24873500517545\\
0.303030303030303	8.48501408791444\\
0.404040404040404	7.62810419183902\\
0.505050505050505	6.27211182019865\\
0.606060606060606	4.76712608622486\\
0.707070707070707	3.30925701855677\\
0.808080808080808	1.99907294684497\\
0.909090909090909	0.879250579676432\\
1.01010101010101	-0.0413999490737453\\
1.11111111111111	-0.772800007821904\\
1.21212121212121	-1.3342342074407\\
1.31313131313131	-1.74869493412785\\
1.41414141414141	-2.03959372519954\\
1.51515151515152	-2.22896159716534\\
1.61616161616162	-2.33656911394331\\
1.71717171717172	-2.37959995455457\\
1.81818181818182	-2.3726447096012\\
1.91919191919192	-2.32786833738851\\
2.02020202020202	-2.25526090296452\\
2.12121212121212	-2.16291735803513\\
2.22222222222222	-2.05731511725016\\
2.32323232323232	-1.94357262051049\\
2.42424242424242	-1.82568095723205\\
2.52525252525253	-1.70670594908193\\
2.62626262626263	-1.58896114054222\\
2.72727272727273	-1.47415377759396\\
2.82828282828283	-1.3635066116235\\
2.92929292929293	-1.25785860023122\\
3.03030303030303	-1.15774751261539\\
3.13131313131313	-1.06347722471251\\
3.23232323232323	-0.975172194311307\\
3.33333333333333	-0.892821290203309\\
3.43434343434343	-0.816312841310279\\
3.53535353535354	-0.745462487056689\\
3.63636363636364	-0.680035155984561\\
3.73737373737374	-0.619762277654211\\
3.83838383838384	-0.564355142290825\\
3.93939393939394	-0.513515160976095\\
4.04040404040404	-0.46694164333147\\
4.14141414141414	-0.424337596307044\\
4.24242424242424	-0.385413953697805\\
4.34343434343434	-0.349892568430173\\
4.44444444444444	-0.317508235887985\\
4.54545454545455	-0.288009964303025\\
4.64646464646465	-0.26116166556695\\
4.74747474747475	-0.23674240507047\\
4.84848484848485	-0.214546320946267\\
4.94949494949495	-0.194382300217747\\
5.05050505050505	-0.176073480866817\\
5.15151515151515	-0.159456633927615\\
5.25252525252525	-0.144381467727552\\
5.35353535353535	-0.130709886787636\\
5.45454545454545	-0.118315230214114\\
5.55555555555556	-0.107081508297353\\
5.65656565656566	-0.0969026511826619\\
5.75757575757576	-0.0876817796478156\\
5.85858585858586	-0.0793305050146721\\
5.95959595959596	-0.0717682628761589\\
6.06060606060606	-0.06492168350382\\
6.16161616161616	-0.0587240004087878\\
6.26262626262626	-0.0531144974743742\\
6.36363636363636	-0.0480379942918722\\
6.46464646464646	-0.0434443687564695\\
6.56565656565657	-0.039288115572166\\
6.66666666666667	-0.035527939036896\\
6.76767676767677	-0.0321263783025283\\
6.86868686868687	-0.029049463205722\\
6.96969696969697	-0.0262663987261633\\
7.07070707070707	-0.0237492761337015\\
7.17171717171717	-0.0214728089236519\\
7.27272727272727	-0.0194140917008053\\
7.37373737373737	-0.0175523802502645\\
7.47474747474747	-0.0158688911214844\\
7.57575757575758	-0.0143466191464887\\
7.67676767676768	-0.0129701714108536\\
7.77777777777778	-0.0117256162941745\\
7.87878787878788	-0.0106003462935228\\
7.97979797979798	-0.00958295343749034\\
8.08080808080808	-0.008663116188869\\
8.18181818181818	-0.00783149682018545\\
8.28282828282828	-0.00707964832780434\\
8.38383838383838	-0.00639993002692369\\
8.48484848484848	-0.00578543104144157\\
8.58585858585859	-0.00522990096940971\\
8.68686868686869	-0.00472768706672082\\
8.78787878787879	-0.00427367734897007\\
8.88888888888889	-0.00386324906429221\\
8.98989898989899	-0.00349222203863477\\
9.09090909090909	-0.00315681643962756\\
9.19191919191919	-0.00285361454619861\\
9.29292929292929	-0.00257952614861715\\
9.39393939393939	-0.00233175723795842\\
9.49494949494949	-0.00210778167532253\\
9.5959595959596	-0.00190531555972657\\
9.6969696969697	-0.00172229403964251\\
9.7979797979798	-0.00155685033687829\\
9.8989898989899	-0.00140729677308631\\
10	-0.00127210760881377\\
};
\addlegendentry{y2}

\end{axis}
\end{tikzpicture}%
    \end{tikzpicture}}
    \caption{Output response of $\bm{H}_1$ to $\bm{u} = \left[2e^{-t}\: 0\right]^T$.}
    \label{Tf1}
\end{figure}
For $\bm{K} = -\bm{I}$, $\bm{KG}$ and $\bm{GK}$ have no RHP Pole-Zero cancellations, thus it is enough to check the stability of one of the closed-loop transfer functions to determine the internal stability of the closed-loop system \cite{Zho99}.

As seen in Figure \ref{Tf1}, $\bm{H}_1$ is stable, thus our closed-loop system will be internally stable as well.
\subsection{Modelling Dynamic Uncertainty}
\begin{figure}[h!]
  \centering
  \tikzstyle{block}     = [draw, rectangle, minimum height=1cm, minimum width=1.6cm]
    \tikzstyle{branch}    = [circle, inner sep=0pt, minimum size=1mm, fill=black, draw=black]
    \tikzstyle{connector} = [->, thick]
    \tikzstyle{dummy}     = [inner sep=0pt, minimum size=0pt]
    \tikzstyle{inout}     = []
    \tikzstyle{sum}       = [circle, inner sep=0pt, minimum size=2mm, draw=black, thick]
    \begin{tikzpicture}[auto, node distance=1.5cm, >=stealth']
    % Nodes
    \node[block] (G) {$\mathbf{G}$};
    \node[sum, left of = G, node distance = 2.5cm] (s1) {};
    \node[sum, right of = G, node distance = 2cm] (s2) {};
    \node[block, left of = s1] (K) {$\mathbf{K}$};
    \node[block, above of = G] (delA) {$\mathbf{\Delta}_A$};
    \node[block, right of = delA, node distance = 3.75cm] (delM) {$\mathbf{\Delta}_M$};
    \node[sum, right of = s2, node distance = 3cm] (s3) {};
    \node[sum, left of = K] (s4) {};
    \node[inout, left of = s4] (w2) {$\mathbf{w}_2$};
    \node[inout, above of = s1] (w1) {$\mathbf{w}_1$};
    \node[branch, right of = s3] (b1) {};
    \node[branch, below of = b1] (b2) {};
    \node[branch, right of = s1, node distance = 1cm] (b3) {};
    \node[branch, right of = G] (b4) {};
    \node[branch, right of = s2, node distance = 0.5cm] (b5) {};
    \node[inout, right of = b1] (y) {$\mathbf{y}$};

    % Connections
    \draw[connector] (b1) -- (y);
    \draw[thick] (b1) -- (b2);
    \draw[connector] (b2) -| (s4);
    \draw[connector] (w2) -- (s4);
    \draw[connector] (s4) -- (K);
    \draw[connector] (K) -- node[] {$\mathbf{u}$} (s1);
    \draw[connector] (w1) -- (s1);
    \draw[connector] (s1) -- (b3);
    \draw[connector] (b3) -- (G);
    %      \draw[connector] (s3) -| node [yshift=0.3cm] {$\mathbf{y}_m$} (s1);
    \draw[connector] (b3) |- node[] {$\mathbf{y}_A$} (delA);
    \draw[connector] (delA) -| node[yshift = 0.2cm, xshift = -0.5cm] {$\mathbf{u}_A$} (b4);
    \draw[connector] (G) -- (b4);
    \draw[connector] (b4) -- (s2);
    \draw[connector] (s2) -- (b5);
    \draw[connector] (b5) -- (s3);
    \draw[connector] (b5) |- node[] {$\mathbf{y}_M$} (delM);
    \draw[connector] (delM) -| node[yshift = 0.2cm, xshift = -0.5cm] {$\mathbf{u}_M$} (s3);
    \draw[connector] (s3) -- (b1);

    \end{tikzpicture}
	\caption{Uncertain System with Additive and Output Multiplicative Uncertainty.}
    \label{fig:blockDiagram2}
\end{figure}

\begin{figure}[h!]
  \centering
  \tikzstyle{block}     = [draw, rectangle, minimum height=1cm, minimum width=1.2cm]
    \tikzstyle{branch}    = [circle, inner sep=0pt, minimum size=1mm, fill=black, draw=black]
    \tikzstyle{connector} = [->, thin]
    \tikzstyle{dummy}     = [inner sep=0pt, minimum size=0pt]
    \tikzstyle{inout}     = []
    \tikzstyle{sum}       = [circle, inner sep=0pt, minimum size=2mm, draw=black, thick]
    \begin{tikzpicture}[auto, node distance=1.5cm, >=stealth']
    % Nodes
      \node[block] (P) {$\bm{P}$};
      \node[block, below of = P] (K) {$\bm{K}$};
      \node[inout, left of = P] (w) {$\bm{w}$};
      \node[inout, right of = P] (z) {$\bm{z}$};
      \node[block, above of = P] (delta) {$\begin{matrix}
          \mathbf{\Delta}_A & \mathbf{O}\\
        \mathbf{O} & \mathbf{\Delta}_M
      \end{matrix}$};
      % Connections
      \draw[->] (K.west) -| ++(-1,1) node [yshift = -0.3cm, xshift = -0.3cm] {$\bm{u}$} |- (P.210);
      \draw[->] (P.-30) -| ++(1,-1) node [yshift = 0.3cm, xshift = 0.3cm] {$\bm{v}$} |- (K.east);
      \draw[->] (delta.west) -| ++(-0.63,-1) node [yshift = 0.2cm, xshift = -0.3cm] {$\bm{u}_\Delta$} |- (P.150);
      \draw[->] (P.30) -| ++(1,1) node [yshift = -0.3cm, xshift = 0.3cm] {$\bm{y}_\Delta$} |- (delta.east);
      \draw[connector] (P.0) -- (z);
	 \draw[connector] (w) -- (P.180);
    \end{tikzpicture}
	  \caption{Generalized Uncertain Closed Loop Configuration with Dynamic Uncertainty.}
    \label{fig:blockDiagram3}
\end{figure}
The block diagram in Figure \ref{fig:blockDiagram2} can be rearranged in as the $\mathbf{M\Delta}$ structure with the perturbation matrix being set as
\begin{align*}
    \mathbf{\Delta} = 
    \begin{bmatrix}
        \mathbf{\Delta}_A & \mathbf{O}\\
        \mathbf{O} & \mathbf{\Delta}_M
    \end{bmatrix}
\end{align*}
Thus, $\mathbf{\Delta}$ is a structured perturbation with a block diagonal structure. 
The uncertainty and the controller can be taken out of the system, as shown in Figure \ref{fig:blockDiagram3}. The generalized plant $\mathbf{P}$ can then be derived by the procedure as described in Chapter-8 of \cite{Sko05}. Assuming,
\begin{align*}
    \bm{z} = \bm{y}\\
    \bm{w} = 
    \begin{bmatrix}
        \bm{w}_1\\
        \bm{w}_2
    \end{bmatrix}
\end{align*}
Our generalized plant will be,
\begin{align*}
    \bm{P} = 
    \begin{bmatrix}
        \bm{O} & \bm{O} & \bm{I} & \bm{O} &\bm{I}\\
        \bm{I} & \bm{O} & \bm{G} & \bm{O} & \bm{G}\\
        \bm{I} & \bm{I} & \bm{G} & \bm{O} & \bm{G}\\
        \bm{I} & \bm{I} & \bm{G} & \bm{I} & \bm{G}
    \end{bmatrix}
\end{align*}

The nominal closed-loop system can be obtained by closing the lower loop using a lower LFT, $$\bm{N} = \mathcal{F}_l(\bm{P},\bm{K})$$

Since, $\bm{M}$ maps $\bm{u}_\Delta$ to $\bm{y}_\Delta$, thus $$\bm{M} = \bm{N}_{11} =
\begin{bmatrix}
    \bm{K}\bm{S} & \bm{K}\bm{S}\\
    \bm{I} + \bm{T} & \bm{T}
\end{bmatrix}$$
where, $\bm{S} = \left(\bm{I} - \bm{GK}\right)^{-1}$ and $\bm{T} = \bm{GK}\left(\bm{I} - \bm{GK}\right)^{-1}$

\begin{figure}[h!]
    \centering
    \scalebox{0.7}{
    \begin{tikzpicture}
        \input{images/q213b}
    \end{tikzpicture}
    }
    \caption{Singular Value Plot for the upper singular values of $\mathbf{M}$.}
    \label{Msing}
\end{figure}

As seen in Figure \ref{Msing}, the $\mathcal{H}_\infty$ norm of $\mathbf{M}$ lies just below $33$, which can be considered the worst-case perturbation ($\gamma = 33$) for the given structured dynamic uncertainty. The frequency at which this worst-case occurs is $2.8313$ rad/s ($\omega^* = 2.8313$).

\subsection{Modelling Perturbation Matrix}
Let
\begin{align*}
    \bm{\Delta}_A = \delta_A\bm{I}\\
    \bm{\Delta}_M = \delta_M\bm{I}
\end{align*}
where $\delta_A, \delta_M \in \mathbb{C}$. If both $\delta_A$ and $\delta_A$ are considered to be \emph{all-pass} elements, such that,
\begin{align*}
    \delta_A = \gamma\left(\frac{s-2}{s+2}\right),
    \delta_M = 0.01\left(\frac{s-2}{s+2}\right)
\end{align*}
\begin{align*}
    \Bar{\sigma}\left(\bm{\Delta}(j\omega^*)\right) = \gamma\\ 
    \|\bm{\Delta}\|_{\infty} = \gamma,
    \bm{\Delta} \in \mathcal{RH}_{\infty}
\end{align*}
    
\begin{figure}[h!]
    \centering
    \scalebox{0.7}{\begin{tikzpicture}
        % This file was created by matlab2tikz.
%
%The latest updates can be retrieved from
%  http://www.mathworks.com/matlabcentral/fileexchange/22022-matlab2tikz-matlab2tikz
%where you can also make suggestions and rate matlab2tikz.
%
\begin{tikzpicture}

\begin{axis}[%
width=4.521in,
height=3.566in,
at={(0.758in,0.481in)},
scale only axis,
xmin=0,
xmax=10,
xlabel style={font=\color{white!15!black}},
xlabel={Time(s)},
ymin=-1.2e+25,
ymax=2e+24,
ylabel style={font=\color{white!15!black}},
ylabel={Amplitude},
axis background/.style={fill=white},
xmajorgrids,
ymajorgrids,
legend style={legend cell align=left, align=left, draw=white!15!black}
]
\addplot [color=red, line width=2.0pt]
  table[row sep=crcr]{%
0	0.970870958345465\\
0.00730994152046784	0.96724940737253\\
0.0146198830409357	0.963408027244316\\
0.0219298245614035	0.959335182128387\\
0.0292397660818713	0.955018632250065\\
0.0365497076023392	0.950445503813924\\
0.043859649122807	0.945602257452964\\
0.0511695906432749	0.940474655134725\\
0.0584795321637427	0.935047725450278\\
0.0657894736842105	0.929305727208498\\
0.0730994152046784	0.923232111254332\\
0.0804093567251462	0.916809480425965\\
0.087719298245614	0.910019547561697\\
0.0950292397660819	0.902843091463187\\
0.10233918128655	0.895259910717245\\
0.109649122807018	0.887248775273756\\
0.116959064327485	0.878787375672491\\
0.124269005847953	0.86985226980645\\
0.131578947368421	0.860418827104108\\
0.138888888888889	0.850461170007384\\
0.146198830409357	0.839952112616309\\
0.153508771929825	0.828863096365317\\
0.160818713450292	0.817164122589691\\
0.16812865497076	0.804823681834042\\
0.175438596491228	0.791808679747704\\
0.182748538011696	0.778084359404654\\
0.190058479532164	0.76361421987788\\
0.197368421052632	0.748359930890156\\
0.204678362573099	0.732281243354787\\
0.211988304093567	0.715335895611128\\
0.219298245614035	0.697479515150503\\
0.226608187134503	0.678665515618571\\
0.233918128654971	0.658844988870094\\
0.241228070175439	0.637966591841599\\
0.248538011695906	0.615976427996387\\
0.255847953216374	0.592817923084827\\
0.263157894736842	0.568431694950848\\
0.27046783625731	0.542755417102892\\
0.277777777777778	0.515723675754419\\
0.285087719298246	0.487267820025239\\
0.292397660818713	0.457315804980488\\
0.299707602339181	0.425792027168928\\
0.307017543859649	0.392617152306467\\
0.314327485380117	0.35770793473418\\
0.321637426900585	0.320977028262819\\
0.328947368421053	0.282332787997665\\
0.33625730994152	0.24167906271858\\
0.343567251461988	0.198914977370291\\
0.350877192982456	0.153934705197148\\
0.358187134502924	0.106627229034869\\
0.365497076023392	0.0568760912490444\\
0.37280701754386	0.00455913178637691\\
0.380116959064327	-0.0504517862202577\\
0.387426900584795	-0.108291063877895\\
0.394736842105263	-0.169099667602936\\
0.402046783625731	-0.23302544447468\\
0.409356725146199	-0.300223452543527\\
0.416666666666667	-0.37085630679558\\
0.423976608187135	-0.445094541508101\\
0.431286549707602	-0.523116989764381\\
0.43859649122807	-0.605111180932339\\
0.445906432748538	-0.691273756948563\\
0.453216374269006	-0.781810908288602\\
0.460526315789474	-0.876938830545233\\
0.467836257309941	-0.976884202579259\\
0.475146198830409	-1.08188468725213\\
0.482456140350877	-1.19218945579656\\
0.489766081871345	-1.30805973693032\\
0.497076023391813	-1.42976939186958\\
0.504385964912281	-1.55760551645188\\
0.511695906432749	-1.69186907163489\\
0.519005847953216	-1.83287554369573\\
0.526315789473684	-1.98095563551705\\
0.533625730994152	-2.13645599041032\\
0.54093567251462	-2.29973994999373\\
0.548245614035088	-2.47118834771277\\
0.555555555555556	-2.6512003396644\\
0.562865497076023	-2.84019427446332\\
0.570175438596491	-3.0386086039685\\
0.577485380116959	-3.24690283677289\\
0.584795321637427	-3.4655585364466\\
0.592105263157895	-3.69508036661628\\
0.599415204678363	-3.93599718505928\\
0.60672514619883	-4.18886318909211\\
0.614035087719298	-4.45425911463774\\
0.621345029239766	-4.73279349146646\\
0.628654970760234	-5.02510395722017\\
0.635964912280702	-5.33185863295029\\
0.64327485380117	-5.65375756302551\\
0.650584795321637	-5.99153422239733\\
0.657894736842105	-6.34595709434886\\
0.665204678362573	-6.71783132199681\\
0.672514619883041	-7.10800043696685\\
0.679824561403509	-7.51734816882035\\
0.687134502923977	-7.94680033897526\\
0.694444444444444	-8.39732684303607\\
0.701754385964912	-8.8699437256282\\
0.70906432748538	-9.3657153520205\\
0.716374269005848	-9.88575668101677\\
0.723684210526316	-10.4312356438031\\
0.730994152046784	-11.0033756336537\\
0.738304093567251	-11.6034581116228\\
0.745614035087719	-12.2328253335866\\
0.752923976608187	-12.8928832042453\\
0.760233918128655	-13.5851042639523\\
0.767543859649123	-14.3110308145089\\
0.774853801169591	-15.0722781903436\\
0.782163742690059	-15.8705381817885\\
0.789473684210526	-16.7075826174775\\
0.796783625730994	-17.5852671132075\\
0.804093567251462	-18.5055349949468\\
0.81140350877193	-19.4704214040221\\
0.818713450292398	-20.482057592888\\
0.826023391812866	-21.5426754202641\\
0.833333333333333	-22.6546120548324\\
0.840643274853801	-23.8203148971042\\
0.847953216374269	-25.042346729509\\
0.855263157894737	-26.3233911052173\\
0.862573099415205	-27.6662579866913\\
0.869883040935672	-29.0738896454598\\
0.87719298245614	-30.5493668351409\\
0.884502923976608	-32.0959152502861\\
0.891812865497076	-33.7169122841951\\
0.899122807017544	-35.4158940994507\\
0.906432748538012	-37.1965630255544\\
0.91374269005848	-39.0627952987003\\
0.921052631578947	-41.0186491594095\\
0.928362573099415	-43.068373324471\\
0.935672514619883	-45.2164158503838\\
0.942982456140351	-47.4674334062782\\
0.950292397660819	-49.826300975123\\
0.957602339181287	-52.2981220028747\\
0.964912280701754	-54.8882390161315\\
0.972222222222222	-57.602244729787\\
0.97953216374269	-60.445993667164\\
0.986842105263158	-63.4256143161316\\
0.994152046783626	-66.5475218457848\\
1.00146198830409	-69.8184314093834\\
1.00877192982456	-73.2453720604225\\
1.01608187134503	-76.8357013099288\\
1.0233918128655	-80.5971203543628\\
1.03070175438596	-84.5376900048404\\
1.03801169590643	-88.6658473497931\\
1.0453216374269	-92.9904231846447\\
1.05263157894737	-97.5206602436163\\
1.05994152046784	-102.266232270368\\
1.0672514619883	-107.237263965858\\
1.07456140350877	-112.444351853552\\
1.08187134502924	-117.898586103933\\
1.08918128654971	-123.611573362184\\
1.09649122807018	-129.595460624898\\
1.10380116959064	-135.862960213778\\
1.11111111111111	-142.427375896437\\
1.11842105263158	-149.302630206737\\
1.12573099415205	-156.503293019427\\
1.13304093567251	-164.044611436408\\
1.14035087719298	-171.942541044473\\
1.14766081871345	-180.213778607169\\
1.15497076023392	-188.875796256223\\
1.16228070175439	-197.946877250963\\
1.16959064327485	-207.446153377295\\
1.17690058479532	-217.393644061004\\
1.18421052631579	-227.810297273582\\
1.19152046783626	-238.718032312329\\
1.19883040935673	-250.139784540155\\
1.20614035087719	-262.099552174438\\
1.21345029239766	-274.622445218305\\
1.22076023391813	-287.734736631961\\
1.2280701754386	-301.463915846129\\
1.23538011695906	-315.838744724261\\
1.24269005847953	-330.889316085054\\
1.25	-346.647114901845\\
1.25730994152047	-363.145082300743\\
1.26461988304094	-380.417682484891\\
1.2719298245614	-398.50097271801\\
1.27923976608187	-417.432676506438\\
1.28654970760234	-437.252260125142\\
1.29385964912281	-458.00101263983\\
1.30116959064327	-479.722129584122\\
1.30847953216374	-502.460800457982\\
1.31578947368421	-526.264300221117\\
1.32309941520468	-551.182084962929\\
1.33040935672515	-577.265891938805\\
1.33771929824561	-604.569844171151\\
1.34502923976608	-633.150559822525\\
1.35233918128655	-663.067266557628\\
1.35964912280702	-694.381921120713\\
1.36695906432749	-727.159334365211\\
1.37426900584795	-761.467301983106\\
1.38157894736842	-797.376741192767\\
1.38888888888889	-834.961833655638\\
1.39619883040936	-874.300174904443\\
1.40350877192982	-915.472930578298\\
1.41081871345029	-958.564999773522\\
1.41812865497076	-1003.66518583283\\
1.42543859649123	-1050.86637491026\\
1.4327485380117	-1100.26572266432\\
1.44005847953216	-1151.96484944782\\
1.44736842105263	-1206.07004437963\\
1.4546783625731	-1262.6924787006\\
1.46198830409357	-1321.94842883462\\
1.46929824561404	-1383.95950959422\\
1.4766081871345	-1448.85291799039\\
1.48391812865497	-1516.76168812684\\
1.49122807017544	-1587.82495768055\\
1.49853801169591	-1662.18824649329\\
1.50584795321637	-1740.00374782226\\
1.51315789473684	-1821.43063282287\\
1.52046783625731	-1906.63536886246\\
1.52777777777778	-1995.79205229084\\
1.53508771929825	-2089.0827563215\\
1.54239766081871	-2186.69789470729\\
1.54970760233918	-2288.83660192468\\
1.55701754385965	-2395.70713061326\\
1.56432748538012	-2507.52726705064\\
1.57163742690058	-2624.52476547806\\
1.57894736842105	-2746.9378021288\\
1.58625730994152	-2875.01544984994\\
1.59356725146199	-3009.01817424786\\
1.60087719298246	-3149.21835233017\\
1.60818713450292	-3295.90081466022\\
1.61549707602339	-3449.36341208623\\
1.62280701754386	-3609.91760815505\\
1.63011695906433	-3777.88909837018\\
1.6374269005848	-3953.61845750614\\
1.64473684210526	-4137.46181624585\\
1.65204678362573	-4329.79156846447\\
1.6593567251462	-4530.9971105429\\
1.66666666666667	-4741.48561415618\\
1.67397660818713	-4961.68283404734\\
1.6812865497076	-5192.03395236481\\
1.68859649122807	-5433.00446121269\\
1.69590643274854	-5685.08108513751\\
1.70321637426901	-5948.77274535205\\
1.71052631578947	-6224.61156757855\\
1.71783625730994	-6513.15393547734\\
1.72514619883041	-6814.98159171609\\
1.73245614035088	-7130.70278882675\\
1.73976608187134	-7460.95349209387\\
1.74707602339181	-7806.39863681886\\
1.75438596491228	-8167.73344240995\\
1.76169590643275	-8545.68478585779\\
1.76900584795322	-8941.01263727147\\
1.77631578947368	-9354.51156027005\\
1.78362573099415	-9787.01228014991\\
1.79093567251462	-10239.3833228797\\
1.79824561403509	-10712.5327281111\\
1.80555555555556	-11207.4098395382\\
1.81286549707602	-11725.0071760848\\
1.82017543859649	-12266.3623875596\\
1.82748538011696	-12832.5602985778\\
1.83479532163743	-13424.7350447215\\
1.84210526315789	-14044.0723050889\\
1.84941520467836	-14691.8116355661\\
1.85672514619883	-15369.2489073527\\
1.8640350877193	-16077.7388554756\\
1.87134502923977	-16818.6977422328\\
1.87865497076023	-17593.606140739\\
1.8859649122807	-18404.0118439682\\
1.89327485380117	-19251.532904936\\
1.90058479532164	-20137.8608139152\\
1.90789473684211	-21064.7638188422\\
1.91520467836257	-22034.090395348\\
1.92251461988304	-23047.772873137\\
1.92982456140351	-24107.8312257355\\
1.93713450292398	-25216.377030949\\
1.94444444444444	-26375.6176096951\\
1.95175438596491	-27587.8603512211\\
1.95906432748538	-28855.5172330764\\
1.96637426900585	-30181.1095445831\\
1.97368421052632	-31567.2728229396\\
1.98099415204678	-33016.7620115014\\
1.98830409356725	-34532.4568502119\\
1.99561403508772	-36117.367508599\\
2.00292397660819	-37774.6404722239\\
2.01023391812865	-39507.5646939515\\
2.01754385964912	-41319.5780219243\\
2.02485380116959	-43214.2739166499\\
2.03216374269006	-45195.4084701712\\
2.03947368421053	-47266.9077408645\\
2.04678362573099	-49432.8754180203\\
2.05409356725146	-51697.6008309926\\
2.06140350877193	-54065.5673183639\\
2.0687134502924	-56541.4609732657\\
2.07602339181287	-59130.1797817125\\
2.08333333333333	-61836.8431715651\\
2.0906432748538	-64666.8019905246\\
2.09795321637427	-67625.648932379\\
2.10526315789474	-70719.2294315883\\
2.1125730994152	-73953.6530471881\\
2.11988304093567	-77335.305357929\\
2.12719298245614	-80870.8603915518\\
2.13450292397661	-84567.2936121187\\
2.14181286549708	-88431.89549039\\
2.14912280701754	-92472.2856833535\\
2.15643274853801	-96696.4278501782\\
2.16374269005848	-101112.645133084\\
2.17105263157895	-105729.636332887\\
2.17836257309942	-110556.492810322\\
2.18567251461988	-115602.716145604\\
2.19298245614035	-120878.236590184\\
2.20029239766082	-126393.432346126\\
2.20760233918129	-132159.149710134\\
2.21491228070175	-138186.724120925\\
2.22222222222222	-144488.002150337\\
2.22953216374269	-151075.364480392\\
2.23684210526316	-157961.749910411\\
2.24415204678363	-165160.680440226\\
2.25146198830409	-172686.287477632\\
2.25877192982456	-180553.33922032\\
2.26608187134503	-188777.269264803\\
2.2733918128655	-197374.206497189\\
2.28070175438596	-206361.006323088\\
2.28801169590643	-215755.283296504\\
2.2953216374269	-225575.445210232\\
2.30263157894737	-235840.728713063\\
2.30994152046784	-246571.236522012\\
2.3172514619883	-257787.976300839\\
2.32456140350877	-269512.901279285\\
2.33187134502924	-281768.952690783\\
2.33918128654971	-294580.104109861\\
2.34649122807018	-307971.40777408\\
2.35380116959064	-321969.042979121\\
2.36111111111111	-336600.366639589\\
2.36842105263158	-351893.966112232\\
2.37573099415205	-367879.714382571\\
2.38304093567251	-384588.827720444\\
2.39035087719298	-402053.925914662\\
2.39766081871345	-420309.095201884\\
2.40497076023392	-439389.954009947\\
2.41228070175439	-459333.721641249\\
2.41959064327485	-480179.290027346\\
2.42690058479532	-501967.29869181\\
2.43421052631579	-524740.213064456\\
2.44152046783626	-548542.406296446\\
2.44883040935673	-573420.2447324\\
2.45614035087719	-599422.177202649\\
2.46345029239766	-626598.828305962\\
2.47076023391813	-655003.095860698\\
2.4780701754386	-684690.252710259\\
2.48538011695906	-715718.053076962\\
2.49269005847953	-748146.843667103\\
2.5	-782039.679739012\\
2.50730994152047	-817462.446355319\\
2.51461988304094	-854483.985050467\\
2.5219298245614	-893176.226154825\\
2.52923976608187	-933614.327027471\\
2.53654970760234	-975876.816460904\\
2.54385964912281	-1020045.74553269\\
2.55116959064327	-1066206.84519127\\
2.55847953216374	-1114449.69087587\\
2.56578947368421	-1164867.87448385\\
2.57309941520468	-1217559.18401289\\
2.58040935672515	-1272625.7912195\\
2.58771929824561	-1330174.44765118\\
2.59502923976608	-1390316.68942492\\
2.60233918128655	-1453169.0511415\\
2.60964912280702	-1518853.28934233\\
2.61695906432749	-1587496.61593372\\
2.62426900584795	-1659231.94202218\\
2.63157894736842	-1734198.13262418\\
2.63888888888889	-1812540.27273447\\
2.64619883040936	-1894409.94525828\\
2.65350877192982	-1979965.52133554\\
2.66081871345029	-2069372.46360844\\
2.66812865497076	-2162803.64300813\\
2.67543859649123	-2260439.66966234\\
2.6827485380117	-2362469.2385517\\
2.69005847953216	-2469089.49057121\\
2.69736842105263	-2580506.38968189\\
2.7046783625731	-2696935.11686819\\
2.71198830409357	-2818600.48164882\\
2.71929824561404	-2945737.35192133\\
2.7266081871345	-3078591.10295583\\
2.73391812865497	-3217418.08638927\\
2.74122807017544	-3362486.12010945\\
2.74853801169591	-3514074.99995754\\
2.75584795321637	-3672477.03421884\\
2.76315789473684	-3837997.601915\\
2.77046783625731	-4010955.7359553\\
2.77777777777778	-4191684.732252\\
2.78508771929825	-4380532.78595355\\
2.79239766081871	-4577863.65600072\\
2.79970760233918	-4784057.35926405\\
2.80701754385965	-4999510.89557713\\
2.81432748538012	-5224639.00503807\\
2.82163742690058	-5459874.95901288\\
2.82894736842105	-5705671.38633765\\
2.83625730994152	-5962501.13628309\\
2.84356725146199	-6230858.1799142\\
2.85087719298246	-6511258.55155017\\
2.85818713450292	-6804241.33210537\\
2.86549707602339	-7110369.67617115\\
2.87280701754386	-7430231.88478059\\
2.88011695906433	-7764442.52588449\\
2.8874269005848	-8113643.60465679\\
2.89473684210526	-8478505.78584144\\
2.90204678362573	-8859729.67045089\\
2.9093567251462	-9258047.12922852\\
2.91666666666667	-9674222.69539456\\
2.92397660818713	-10109055.0193065\\
2.9312865497076	-10563378.3877813\\
2.93859649122807	-11038064.3109494\\
2.94590643274854	-11534023.1796359\\
2.95321637426901	-12052205.9963994\\
2.96052631578947	-12593606.1834953\\
2.96783625730994	-13159261.4711761\\
2.97514619883041	-13750255.8698938\\
2.98245614035088	-14367721.730124\\
2.98976608187135	-15012841.8936988\\
2.99707602339181	-15686851.9407084\\
3.00438596491228	-16391042.5362066\\
3.01169590643275	-17126761.881149\\
3.01900584795322	-17895418.2721836\\
3.02631578947368	-18698482.7751214\\
3.03362573099415	-19537492.0171265\\
3.04093567251462	-20414051.102889\\
3.04824561403509	-21329836.6602772\\
3.05555555555556	-22286600.0212073\\
3.06286549707602	-23286170.5437265\\
3.07017543859649	-24330459.0815648\\
3.07748538011696	-25421461.607695\\
3.08479532163743	-26561262.9987219\\
3.09210526315789	-27752040.9872313\\
3.09941520467836	-28996070.2895378\\
3.10672514619883	-30295726.9166053\\
3.1140350877193	-31653492.6762544\\
3.12134502923977	-33071959.8751308\\
3.12865497076023	-34553836.2292846\\
3.1359649122807	-36101949.9926011\\
3.14327485380117	-37719255.3127319\\
3.15058479532164	-39408837.8246025\\
3.15789473684211	-41173920.4920195\\
3.16520467836257	-43017869.7083618\\
3.17251461988304	-44944201.667832\\
3.17982456140351	-46956589.0192451\\
3.18713450292398	-49058867.8148671\\
3.19444444444444	-51255044.7673644\\
3.20175438596491	-53549304.8285066\\
3.20906432748538	-55946019.1038646\\
3.21637426900585	-58449753.1183781\\
3.22368421052632	-61065275.4483233\\
3.23099415204678	-63797566.7358978\\
3.23830409356725	-66651829.1033569\\
3.24561403508772	-69633495.9843828\\
3.25292397660819	-72748242.3911519\\
3.26023391812865	-76001995.636379\\
3.26754385964912	-79400946.5304707\\
3.27485380116959	-82951561.0748085\\
3.28216374269006	-86660592.6731124\\
3.28947368421053	-90535094.883805\\
3.29678362573099	-94582434.737307\\
3.30409356725146	-98810306.6432568\\
3.31140350877193	-103226746.913744\\
3.3187134502924	-107840148.929806\\
3.32602339181287	-112659278.979635\\
3.33333333333333	-117693292.798207\\
3.3406432748538	-122951752.839339\\
3.34795321637427	-128444646.31258\\
3.35526315789474	-134182404.018735\\
3.3625730994152	-140175920.019348\\
3.36988304093567	-146436572.177012\\
3.37719298245614	-152976243.604993\\
3.38450292397661	-159807345.066387\\
3.39181286549708	-166942838.364768\\
3.39912280701754	-174396260.770149\\
3.40643274853801	-182181750.526044\\
3.41374269005848	-190314073.48537\\
3.42105263157895	-198808650.925124\\
3.42836257309942	-207681588.591886\\
3.43567251461988	-216949707.03257\\
3.44298245614035	-226630573.267199\\
3.45029239766082	-236742533.862996\\
3.45760233918129	-247304749.471728\\
3.46491228070175	-258337230.894921\\
3.47222222222222	-269860876.74447\\
3.47953216374269	-281897512.769098\\
3.48684210526316	-294469932.920271\\
3.49415204678363	-307601942.234384\\
3.50146198830409	-321318401.611458\\
3.50877192982456	-335645274.574083\\
3.51608187134503	-350609676.094098\\
3.5233918128655	-366239923.578283\\
3.53070175438596	-382565590.108435\\
3.53801169590643	-399617560.035351\\
3.5453216374269	-417428087.03067\\
3.55263157894737	-436030854.705084\\
3.55994152046784	-455461039.906216\\
3.5672514619883	-475755378.81449\\
3.57456140350877	-496952235.960475\\
3.58187134502924	-519091676.292704\\
3.58918128654971	-542215540.430577\\
3.59649122807018	-566367523.242972\\
3.60380116959064	-591593255.899314\\
3.61111111111111	-617940391.546375\\
3.61842105263158	-645458694.770804\\
3.62573099415205	-674200135.014456\\
3.63304093567251	-704218984.116939\\
3.64035087719298	-735571918.167484\\
3.64766081871345	-768318123.856301\\
3.65497076023392	-802519409.523914\\
3.66228070175439	-838240321.115746\\
3.66959064327485	-875548263.25837\\
3.67690058479532	-914513625.683341\\
3.68421052631579	-955209915.234505\\
3.69152046783626	-997713893.70508\\
3.69883040935673	-1042105721.76162\\
3.70614035087719	-1088469109.22336\\
3.71345029239766	-1136891471.97715\\
3.72076023391813	-1187464095.82078\\
3.7280701754386	-1240282307.53999\\
3.73538011695906	-1295445653.53836\\
3.74269005847953	-1353058086.35298\\
3.75	-1413228159.40371\\
3.75730994152047	-1476069230.33887\\
3.76461988304094	-1541699673.35663\\
3.7719298245614	-1610243100.89751\\
3.77923976608187	-1681828595.1213\\
3.78654970760234	-1756590949.59961\\
3.79385964912281	-1834670921.67445\\
3.80116959064327	-1916215495.95281\\
3.80847953216374	-2001378159.42822\\
3.81578947368421	-2090319188.74163\\
3.82309941520468	-2183205950.11664\\
3.83040935672515	-2280213212.52763\\
3.83771929824561	-2381523474.68384\\
3.84502923976608	-2487327306.43838\\
3.85233918128655	-2597823705.25754\\
3.85964912280702	-2713220468.41413\\
3.86695906432749	-2833734581.59762\\
3.87426900584795	-2959592624.66421\\
3.88157894736842	-3091031195.28213\\
3.88888888888889	-3228297351.26032\\
3.89619883040936	-3371649072.38361\\
3.90350877192982	-3521355742.61353\\
3.91081871345029	-3677698653.55197\\
3.91812865497076	-3840971530.10397\\
3.92543859649123	-4011481079.31748\\
3.9327485380117	-4189547563.42085\\
3.94005847953216	-4375505398.12356\\
3.94736842105263	-4569703777.29278\\
3.9546783625731	-4772507325.16719\\
3.96198830409357	-4984296777.32057\\
3.96929824561404	-5205469691.64114\\
3.9766081871345	-5436441190.64798\\
3.98391812865497	-5677644736.52443\\
3.99122807017544	-5929532940.30874\\
3.99853801169591	-6192578406.74563\\
4.00584795321637	-6467274616.36859\\
4.01315789473684	-6754136846.45189\\
4.02046783625731	-7053703132.54304\\
4.02777777777778	-7366535272.36207\\
4.03508771929825	-7693219873.93214\\
4.04239766081871	-8034369449.88825\\
4.04970760233918	-8390623559.99632\\
4.05701754385965	-8762650004.00424\\
4.06432748538012	-9151146067.03979\\
4.07163742690059	-9556839819.86764\\
4.07894736842105	-9980491476.4194\\
4.08625730994152	-10422894811.1167\\
4.09356725146199	-10884878638.618\\
4.10087719298246	-11367308358.7359\\
4.10818713450292	-11871087569.3912\\
4.11549707602339	-12397159750.5981\\
4.12280701754386	-12946510022.604\\
4.13011695906433	-13520166981.4468\\
4.1374269005848	-14119204615.3347\\
4.14473684210526	-14744744305.4028\\
4.15204678362573	-15397956914.5592\\
4.1593567251462	-16080064968.2937\\
4.16666666666667	-16792344931.4937\\
4.17397660818713	-17536129585.4899\\
4.1812865497076	-18312810509.7393\\
4.18859649122807	-19123840672.7468\\
4.19590643274854	-19970737137.0291\\
4.20321637426901	-20855083883.1347\\
4.21052631578947	-21778534757.9557\\
4.21783625730994	-22742816552.7959\\
4.22514619883041	-23749732216.899\\
4.23245614035088	-24801164212.3937\\
4.23976608187134	-25899078016.8717\\
4.24707602339181	-27045525780.0881\\
4.25438596491228	-28242650141.5606\\
4.26169590643275	-29492688216.1387\\
4.26900584795322	-30797975754.9258\\
4.27631578947368	-32160951489.2635\\
4.28362573099415	-33584161665.8208\\
4.29093567251462	-35070264781.1899\\
4.29824561403509	-36622036524.7548\\
4.30555555555556	-38242374938.9857\\
4.31286549707602	-39934305806.7146\\
4.32017543859649	-41700988275.3651\\
4.32748538011696	-43545720728.549\\
4.33479532163743	-45471946915.8984\\
4.34210526315789	-47483262352.48\\
4.34941520467836	-49583420999.6349\\
4.35672514619883	-51776342239.6091\\
4.3640350877193	-54066118156.8807\\
4.37134502923977	-56457021139.6573\\
4.37865497076023	-58953511815.6091\\
4.3859649122807	-61560247336.518\\
4.39327485380117	-64282090027.1707\\
4.40058479532164	-67124116414.4939\\
4.40789473684211	-70091626653.6326\\
4.41520467836257	-73190154368.4055\\
4.42251461988304	-76425476924.3367\\
4.42982456140351	-79803626153.2605\\
4.43713450292398	-83330899549.3311\\
4.44444444444444	-87013871957.1376\\
4.45175438596491	-90859407773.5345\\
4.45906432748538	-94874673685.744\\
4.46637426900585	-99067151969.2787\\
4.47368421052632	-103444654370.263\\
4.48099415204678	-108015336597.816\\
4.48830409356725	-112787713453.267\\
4.49561403508772	-117770674624.186\\
4.50292397660819	-122973501172.389\\
4.51023391812866	-128405882746.397\\
4.51754385964912	-134077935550.154\\
4.52485380116959	-140000221101.184\\
4.53216374269006	-146183765812.859\\
4.53947368421053	-152640081436.937\\
4.54678362573099	-159381186404.129\\
4.55409356725146	-166419628102.119\\
4.56140350877193	-173768506132.165\\
4.5687134502924	-181441496587.239\\
4.57602339181287	-189452877396.534\\
4.58333333333333	-197817554783.129\\
4.5906432748538	-206551090883.66\\
4.59795321637427	-215669732580.989\\
4.60526315789474	-225190441603.1\\
4.6125730994152	-235130925943.76\\
4.61988304093567	-245509672662.969\\
4.62719298245614	-256345982127.706\\
4.63450292397661	-267660003756.175\\
4.64181286549708	-279472773331.516\\
4.64912280701754	-291806251953.81\\
4.65643274853801	-304683366702.276\\
4.66374269005848	-318128053082.653\\
4.67105263157895	-332165299338.084\\
4.67836257309941	-346821192705.253\\
4.68567251461988	-362122967701.066\\
4.69298245614035	-378099056528.971\\
4.70029239766082	-394779141697.855\\
4.70760233918129	-412194210950.57\\
4.71491228070175	-430376614603.367\\
4.72222222222222	-449360125401.974\\
4.72953216374269	-469180001004.678\\
4.73684210526316	-489873049207.603\\
4.74415204678363	-511477696032.433\\
4.75146198830409	-534034056802.088\\
4.75877192982456	-557584010335.367\\
4.76608187134503	-582171276397.316\\
4.7733918128655	-607841496548.041\\
4.78070175438596	-634642318538.996\\
4.78801169590643	-662623484412.229\\
4.7953216374269	-691836922464.963\\
4.80263157894737	-722336843248.916\\
4.80994152046784	-754179839781.27\\
4.8172514619883	-787424992151.869\\
4.82456140350877	-822133976719.371\\
4.83187134502924	-858371180097.49\\
4.83918128654971	-896203818141.275\\
4.84649122807018	-935702060152.577\\
4.85380116959064	-976939158533.45\\
4.86111111111111	-1019991584126.24\\
4.86842105263158	-1064939167489.6\\
4.87573099415205	-1111865246370.5\\
4.88304093567251	-1160856819643.89\\
4.89035087719298	-1212004708003.22\\
4.89766081871345	-1265403721697.89\\
4.90497076023392	-1321152835626.1\\
4.91228070175439	-1379355372105.78\\
4.91959064327485	-1440119191659.54\\
4.92690058479532	-1503556892165.25\\
4.93421052631579	-1569786016738.35\\
4.94152046783626	-1638929270728.68\\
4.94883040935673	-1711114748231.08\\
4.95614035087719	-1786476168526.47\\
4.96345029239766	-1865153122888.51\\
4.97076023391813	-1947291332209.8\\
4.9780701754386	-2033042915921.75\\
4.98538011695906	-2122566672702.52\\
4.99269005847953	-2216028373489.66\\
5	-2313601067336.17\\
5.00730994152047	-2415465400672.6\\
5.01461988304094	-2521809950562.28\\
5.0219298245614	-2632831572562.6\\
5.02923976608187	-2748735763831.86\\
5.03654970760234	-2869737042149.49\\
5.04385964912281	-2996059341546.4\\
5.05116959064327	-3127936425272.82\\
5.05847953216374	-3265612316862.87\\
5.06578947368421	-3409341750088.28\\
5.07309941520468	-3559390638628.23\\
5.08040935672515	-3716036566318.65\\
5.08771929824561	-3879569298882.09\\
5.09502923976608	-4050291318078.44\\
5.10233918128655	-4228518379258.23\\
5.10964912280702	-4414580093343.08\\
5.11695906432749	-4608820534302.54\\
5.12426900584795	-4811598873243.6\\
5.13157894736842	-5023290040277.73\\
5.13888888888889	-5244285415381.55\\
5.14619883040936	-5474993549520.08\\
5.15350877192982	-5715840917357.25\\
5.16081871345029	-5967272702936.41\\
5.16812865497076	-6229753619773.63\\
5.17543859649123	-6503768766870.19\\
5.1827485380117	-6789824522216.13\\
5.19005847953216	-7088449475425.82\\
5.19736842105263	-7400195401217.91\\
5.2046783625731	-7725638275527.31\\
5.21198830409357	-8065379336114.68\\
5.21929824561404	-8420046189620.71\\
5.2266081871345	-8790293967097.45\\
5.23391812865497	-9176806530138.07\\
5.24122807017544	-9580297729818.86\\
5.24853801169591	-10001512720764.4\\
5.25584795321637	-10441229332747.5\\
5.26315789473684	-10900259502341.6\\
5.27046783625731	-11379450767252.5\\
5.27777777777778	-11879687826072\\
5.28508771929825	-12401894166315.1\\
5.29239766081871	-12947033763728.6\\
5.29970760233918	-13516112855988.5\\
5.30701754385965	-14110181794040.3\\
5.31432748538012	-14730336974478.4\\
5.32163742690059	-15377722856510.7\\
5.32894736842105	-16053534067205.8\\
5.33625730994152	-16759017598886.7\\
5.34356725146199	-17495475102700.2\\
5.35087719298246	-18264265282568.1\\
5.35818713450292	-19066806393911.4\\
5.36549707602339	-19904578851729\\
5.37280701754386	-20779127952814.3\\
5.38011695906433	-21692066717099.4\\
5.3874269005848	-22645078853339\\
5.39473684210526	-23639921854569.1\\
5.40204678362573	-24678430229016.3\\
5.4093567251462	-25762518872381.4\\
5.41666666666667	-26894186587677.6\\
5.42397660818713	-28075519759075.5\\
5.4312865497076	-29308696186490.3\\
5.43859649122807	-30595989087936.5\\
5.44590643274854	-31939771276987.4\\
5.45321637426901	-33342519522994.4\\
5.46052631578947	-34806819102055.2\\
5.46783625730994	-36335368547071.7\\
5.47514619883041	-37930984605600.2\\
5.48245614035088	-39596607414577.1\\
5.48976608187134	-41335305901402.4\\
5.49707602339181	-43150283421273.7\\
5.50438596491228	-45044883641099.8\\
5.51169590643275	-47022596680769.9\\
5.51900584795322	-49087065523029.5\\
5.52631578947368	-51242092703702.1\\
5.53362573099415	-53491647294509.5\\
5.54093567251462	-55839872191280.5\\
5.54824561403509	-58291091720894\\
5.55555555555556	-60849819580886.5\\
5.56286549707602	-63520767126262.1\\
5.57017543859649	-66308852018679.8\\
5.57748538011696	-69219207253851.4\\
5.58479532163743	-72257190583680.5\\
5.59210526315789	-75428394350389.8\\
5.59941520467836	-78738655750640.6\\
5.60672514619883	-82194067548432.7\\
5.6140350877193	-85800989256394.2\\
5.62134502923977	-89566058805927\\
5.62865497076023	-93496204727567.5\\
5.6359649122807	-97598658863853.6\\
5.64327485380117	-101880969637964\\
5.65058479532164	-106351015902410\\
5.65789473684211	-111017021393124\\
5.66520467836257	-115887569815389\\
5.67251461988304	-120971620589212\\
5.67982456140351	-126278525282949\\
5.68713450292398	-131818044765254\\
5.69444444444444	-137600367106714\\
5.70175438596491	-143636126263932\\
5.70906432748538	-149936421580234\\
5.71637426900585	-156512838138656\\
5.72368421052632	-163377468004455\\
5.73099415204678	-170542932395979\\
5.73830409356725	-178022404824462\\
5.74561403508772	-185829635245039\\
5.75292397660819	-193978975263159\\
5.76023391812866	-202485404442488\\
5.76754385964912	-211364557762397\\
5.77485380116959	-220632754275249\\
5.78216374269006	-230307027015864\\
5.78947368421053	-240405154217860\\
5.79678362573099	-250945691893922\\
5.80409356725146	-261948007839569\\
5.81140350877193	-273432317122563\\
5.8187134502924	-285419719122838\\
5.82602339181287	-297932236190651\\
5.83333333333333	-310992853993597\\
5.8406432748538	-324625563626236\\
5.84795321637427	-338855405559286\\
5.85526315789474	-353708515508683\\
5.8625730994152	-369212172308347\\
5.86988304093567	-385394847874101\\
5.87719298245614	-402286259350055\\
5.88450292397661	-419917423532717\\
5.89181286549708	-438320713672268\\
5.89912280701754	-457529918754762\\
5.90643274853801	-477580305373552\\
5.91374269005848	-498508682302960\\
5.92105263157895	-520353467892144\\
5.92836257309941	-543154760402245\\
5.93567251461988	-566954411415303\\
5.94298245614035	-591796102448990\\
5.95029239766082	-617725424917086\\
5.95760233918129	-644789963581724\\
5.96491228070175	-673039383649782\\
5.97222222222222	-702525521672475\\
5.97953216374269	-733302480414107\\
5.98684210526316	-765426727863205\\
5.99415204678363	-798957200566801\\
6.00146198830409	-833955411476509\\
6.00877192982456	-870485562503294\\
6.01608187134503	-908614661986366\\
6.0233918128655	-948412647290671\\
6.03070175438596	-989952512756716\\
6.03801169590643	-1.03331044323629e+15\\
6.0453216374269	-1.0785659534578e+15\\
6.05263157894737	-1.12580203347554e+15\\
6.05994152046784	-1.17510530046841e+15\\
6.0672514619883	-1.226566157165e+15\\
6.07456140350877	-1.28027895718422e+15\\
6.08187134502924	-1.33634217759312e+15\\
6.08918128654971	-1.39485859899676e+15\\
6.09649122807018	-1.45593549348883e+15\\
6.10380116959064	-1.51968482080568e+15\\
6.11111111111111	-1.58622343304185e+15\\
6.11842105263158	-1.65567328830055e+15\\
6.12573099415205	-1.72816167366864e+15\\
6.13304093567251	-1.80382143792304e+15\\
6.14035087719298	-1.88279123439304e+15\\
6.14766081871345	-1.9652157744213e+15\\
6.15497076023392	-2.05124609188605e+15\\
6.16228070175439	-2.14103981926677e+15\\
6.16959064327485	-2.23476147575693e+15\\
6.17690058479532	-2.33258276794908e+15\\
6.18421052631579	-2.43468290364079e+15\\
6.19152046783626	-2.54124891933341e+15\\
6.19883040935673	-2.65247602202111e+15\\
6.20614035087719	-2.76856794589313e+15\\
6.21345029239766	-2.88973732459976e+15\\
6.22076023391813	-3.01620607976067e+15\\
6.2280701754386	-3.14820582642393e+15\\
6.23538011695906	-3.28597829621479e+15\\
6.24269005847953	-3.42977577894573e+15\\
6.25	-3.57986158349257e+15\\
6.25730994152047	-3.73651051877688e+15\\
6.26461988304094	-3.90000939573131e+15\\
6.2719298245614	-4.07065755116271e+15\\
6.27923976608187	-4.24876739446774e+15\\
6.28654970760234	-4.43466497819751e+15\\
6.29385964912281	-4.62869059351082e+15\\
6.30116959064327	-4.8311993916012e+15\\
6.30847953216374	-5.04256203223014e+15\\
6.31578947368421	-5.26316536054813e+15\\
6.32309941520468	-5.49341311343679e+15\\
6.33040935672515	-5.73372665665906e+15\\
6.33771929824561	-5.98454575416028e+15\\
6.34502923976608	-6.24632937092198e+15\\
6.35233918128655	-6.51955651083066e+15\\
6.35964912280702	-6.80472709108819e+15\\
6.36695906432749	-7.10236285475628e+15\\
6.37426900584795	-7.41300832309763e+15\\
6.38157894736842	-7.73723178944808e+15\\
6.38888888888889	-8.07562635643007e+15\\
6.39619883040936	-8.42881101839651e+15\\
6.40350877192982	-8.79743179107632e+15\\
6.41081871345029	-9.18216289047892e+15\\
6.41812865497076	-9.58370796320453e+15\\
6.42543859649123	-1.00028013704006e+16\\
6.4327485380117	-1.04402095277022e+16\\
6.44005847953216	-1.08967323035969e+16\\
6.44736842105263	-1.13732044787585e+16\\
6.4546783625731	-1.18704972690091e+16\\
6.46198830409357	-1.23895199146793e+16\\
6.46929824561404	-1.29312213392631e+16\\
6.4766081871345	-1.34965918803839e+16\\
6.48391812865497	-1.40866650962252e+16\\
6.49122807017544	-1.47025196507124e+16\\
6.49853801169591	-1.53452812808786e+16\\
6.50584795321637	-1.60161248499936e+16\\
6.51315789473684	-1.67162764901949e+16\\
6.52046783625731	-1.74470158385189e+16\\
6.52777777777778	-1.82096783704018e+16\\
6.53508771929825	-1.9005657834898e+16\\
6.54239766081871	-1.98364087960456e+16\\
6.54970760233918	-2.07034492850056e+16\\
6.55701754385965	-2.1608363567799e+16\\
6.56432748538012	-2.25528050336795e+16\\
6.57163742690059	-2.35384992093949e+16\\
6.57894736842105	-2.45672469048244e+16\\
6.58625730994152	-2.56409274957115e+16\\
6.59356725146199	-2.67615023494672e+16\\
6.60087719298246	-2.79310184002746e+16\\
6.60818713450292	-2.91516118799987e+16\\
6.61549707602339	-3.04255122116874e+16\\
6.62280701754386	-3.17550460727472e+16\\
6.63011695906433	-3.31426416351831e+16\\
6.6374269005848	-3.45908329906152e+16\\
6.64473684210526	-3.61022647681203e+16\\
6.65204678362573	-3.76796969532968e+16\\
6.6593567251462	-3.93260099173163e+16\\
6.66666666666667	-4.10442096651089e+16\\
6.67397660818713	-4.28374333122243e+16\\
6.6812865497076	-4.47089548003287e+16\\
6.68859649122807	-4.66621908617306e+16\\
6.69590643274854	-4.87007072437786e+16\\
6.70321637426901	-5.08282252044513e+16\\
6.71052631578947	-5.30486282909465e+16\\
6.71783625730994	-5.53659694135944e+16\\
6.72514619883041	-5.77844782279557e+16\\
6.73245614035088	-6.03085688385235e+16\\
6.73976608187134	-6.29428478380336e+16\\
6.74707602339181	-6.56921226969967e+16\\
6.75438596491228	-6.85614105187026e+16\\
6.76169590643275	-7.15559471756083e+16\\
6.76900584795322	-7.46811968437184e+16\\
6.77631578947368	-7.79428619522837e+16\\
6.78362573099415	-8.13468935669029e+16\\
6.79093567251462	-8.48995022248956e+16\\
6.79824561403509	-8.86071692426385e+16\\
6.80555555555556	-9.24766585154134e+16\\
6.81286549707602	-9.65150288312058e+16\\
6.82017543859649	-1.00729646720834e+17\\
6.82748538011696	-1.05128199867751e+17\\
6.83479532163743	-1.09718711101896e+17\\
6.84210526315789	-1.14509553002999e+17\\
6.84941520467836	-1.19509463139897e+17\\
6.85672514619883	-1.24727559973527e+17\\
6.8640350877193	-1.30173359452492e+17\\
6.87134502923977	-1.35856792331351e+17\\
6.87865497076023	-1.41788222243089e+17\\
6.8859649122807	-1.47978464558598e+17\\
6.89327485380117	-1.54438806067418e+17\\
6.90058479532164	-1.61181025515494e+17\\
6.90789473684211	-1.68217415037246e+17\\
6.91520467836257	-1.75560802520874e+17\\
6.92251461988304	-1.83224574947527e+17\\
6.92982456140351	-1.91222702746704e+17\\
6.93713450292398	-1.99569765212138e+17\\
6.94444444444444	-2.08280977024294e+17\\
6.95175438596491	-2.17372215927659e+17\\
6.95906432748538	-2.26860051613076e+17\\
6.96637426900585	-2.36761775857556e+17\\
6.97368421052632	-2.47095433976302e+17\\
6.98099415204678	-2.57879857644053e+17\\
6.98830409356725	-2.69134699145313e+17\\
6.99561403508772	-2.80880467115675e+17\\
7.00292397660819	-2.93138563839108e+17\\
7.01023391812866	-3.0593132416891e+17\\
7.01754385964912	-3.19282056142994e+17\\
7.02485380116959	-3.33215083367222e+17\\
7.03216374269006	-3.47755789243717e+17\\
7.03947368421053	-3.62930663124439e+17\\
7.04678362573099	-3.78767348473787e+17\\
7.05409356725146	-3.95294693127642e+17\\
7.06140350877193	-4.1254280174007e+17\\
7.0687134502924	-4.30543090512851e+17\\
7.07602339181287	-4.49328344307174e+17\\
7.08333333333333	-4.68932776241128e+17\\
7.0906432748538	-4.89392089881129e+17\\
7.09795321637427	-5.10743544140151e+17\\
7.10526315789474	-5.330260210005e+17\\
7.1125730994152	-5.56280096184022e+17\\
7.11988304093567	-5.80548112897959e+17\\
7.12719298245614	-6.0587425879026e+17\\
7.13450292397661	-6.32304646253945e+17\\
7.14181286549708	-6.59887396226231e+17\\
7.14912280701754	-6.88672725634425e+17\\
7.15643274853801	-7.18713038647219e+17\\
7.16374269005848	-7.50063021896912e+17\\
7.17105263157895	-7.82779743845306e+17\\
7.17836257309941	-8.16922758473475e+17\\
7.18567251461988	-8.52554213483519e+17\\
7.19298245614035	-8.89738963208519e+17\\
7.20029239766082	-9.28544686435497e+17\\
7.20760233918129	-9.69042009355063e+17\\
7.21491228070175	-1.01130463386071e+18\\
7.22222222222222	-1.05540947143043e+18\\
7.22953216374269	-1.10143678283351e+18\\
7.23684210526316	-1.14947032391562e+18\\
7.24415204678363	-1.19959749772692e+18\\
7.25146198830409	-1.25190951326864e+18\\
7.25877192982456	-1.3065015511462e+18\\
7.26608187134503	-1.36347293642924e+18\\
7.2733918128655	-1.42292731903172e+18\\
7.28070175438596	-1.48497286193949e+18\\
7.28801169590643	-1.54972243762619e+18\\
7.2953216374269	-1.61729383301388e+18\\
7.30263157894737	-1.68780996334969e+18\\
7.30994152046784	-1.76139909538642e+18\\
7.3172514619883	-1.83819508027147e+18\\
7.32456140350877	-1.91833759656641e+18\\
7.33187134502924	-2.00197240383745e+18\\
7.33918128654971	-2.08925160727674e+18\\
7.34649122807018	-2.1803339338338e+18\\
7.35380116959064	-2.27538502035786e+18\\
7.36111111111111	-2.37457771427296e+18\\
7.36842105263158	-2.47809238733108e+18\\
7.37573099415205	-2.58611726301162e+18\\
7.38304093567251	-2.69884875816063e+18\\
7.39035087719298	-2.81649183948887e+18\\
7.39766081871345	-2.93926039557467e+18\\
7.40497076023392	-3.06737762504567e+18\\
7.41228070175439	-3.20107644164276e+18\\
7.41959064327485	-3.34059989690011e+18\\
7.42690058479532	-3.48620162120724e+18\\
7.43421052631579	-3.63814628405202e+18\\
7.44152046783626	-3.79671007427858e+18\\
7.44883040935673	-3.96218120123e+18\\
7.45614035087719	-4.13486041768388e+18\\
7.46345029239766	-4.31506156552777e+18\\
7.47076023391813	-4.50311214516325e+18\\
7.4780701754386	-4.69935390966974e+18\\
7.48538011695906	-4.90414348480459e+18\\
7.49269005847953	-5.11785301596212e+18\\
7.5	-5.34087084326368e+18\\
7.50730994152047	-5.57360220600117e+18\\
7.51461988304094	-5.81646997770997e+18\\
7.5219298245614	-6.06991543320257e+18\\
7.52923976608187	-6.33439904895176e+18\\
7.53654970760234	-6.61040133827309e+18\\
7.54385964912281	-6.89842372281877e+18\\
7.55116959064327	-7.19898944196125e+18\\
7.55847953216374	-7.51264450171308e+18\\
7.56578947368421	-7.83995866490128e+18\\
7.57309941520468	-8.18152648438906e+18\\
7.58040935672515	-8.53796838121554e+18\\
7.58771929824561	-8.90993176960563e+18\\
7.59502923976608	-9.29809223088669e+18\\
7.60233918128655	-9.70315473843729e+18\\
7.60964912280702	-1.01258549358857e+19\\
7.61695906432749	-1.05669604708719e+19\\
7.62426900584795	-1.10272723867879e+19\\
7.63157894736842	-1.15076265750148e+19\\
7.63888888888889	-1.20088952902867e+19\\
7.64619883040936	-1.25319887319228e+19\\
7.65350877192982	-1.30778566937909e+19\\
7.66081871345029	-1.36474902859886e+19\\
7.66812865497076	-1.4241923731357e+19\\
7.67543859649123	-1.48622362400802e+19\\
7.6827485380117	-1.55095539657618e+19\\
7.69005847953216	-1.61850520465187e+19\\
7.69736842105263	-1.68899567347866e+19\\
7.7046783625731	-1.76255476196903e+19\\
7.71198830409357	-1.83931599460009e+19\\
7.71929824561404	-1.91941870338766e+19\\
7.7266081871345	-2.00300828037637e+19\\
7.73391812865497	-2.09023644110289e+19\\
7.74122807017544	-2.18126149950877e+19\\
7.74853801169591	-2.27624865480041e+19\\
7.75584795321637	-2.37537029077506e+19\\
7.76315789473684	-2.47880628815454e+19\\
7.77046783625731	-2.58674435049153e+19\\
7.77777777777778	-2.6993803442382e+19\\
7.78508771929825	-2.81691865359225e+19\\
7.79239766081871	-2.93957255076231e+19\\
7.79970760233918	-3.0675645823225e+19\\
7.80701754385965	-3.20112697235494e+19\\
7.81432748538012	-3.34050204310946e+19\\
7.82163742690059	-3.48594265394142e+19\\
7.82894736842105	-3.63771265932147e+19\\
7.83625730994152	-3.79608738674558e+19\\
7.84356725146199	-3.9613541354099e+19\\
7.85087719298246	-4.13381269655204e+19\\
7.85818713450292	-4.31377589639997e+19\\
7.86549707602339	-4.50157016271042e+19\\
7.87280701754386	-4.6975361159212e+19\\
7.88011695906433	-4.90202918598653e+19\\
7.8874269005848	-5.11542025601065e+19\\
7.89473684210526	-5.33809633384377e+19\\
7.90204678362573	-5.57046125285442e+19\\
7.9093567251462	-5.81293640314548e+19\\
7.91666666666667	-6.06596149453596e+19\\
7.92397660818713	-6.32999535268783e+19\\
7.9312865497076	-6.60551674981746e+19\\
7.93859649122807	-6.89302527149342e+19\\
7.94590643274854	-7.19304222108764e+19\\
7.95321637426901	-7.50611156351504e+19\\
7.96052631578947	-7.8328009099677e+19\\
7.96783625730994	-8.17370254542353e+19\\
7.97514619883041	-8.529434500787e+19\\
7.98245614035088	-8.90064167159967e+19\\
7.98976608187134	-9.28799698534286e+19\\
7.99707602339181	-9.69220261944217e+19\\
8.00438596491228	-1.01139912721754e+20\\
8.01169590643275	-1.05541274887807e+20\\
8.01900584795322	-1.10134090451623e+20\\
8.02631578947368	-1.14926683916931e+20\\
8.03362573099415	-1.19927741597255e+20\\
8.04093567251462	-1.25146327335307e+20\\
8.04824561403509	-1.30591898905093e+20\\
8.05555555555556	-1.36274325126357e+20\\
8.06286549707602	-1.42203903722288e+20\\
8.07017543859649	-1.48391379952777e+20\\
8.07748538011696	-1.54847966056868e+20\\
8.08479532163743	-1.61585361539549e+20\\
8.0921052631579	-1.68615774339535e+20\\
8.09941520467836	-1.75951942916279e+20\\
8.10672514619883	-1.83607159296136e+20\\
8.1140350877193	-1.91595293119285e+20\\
8.12134502923977	-1.99930816730891e+20\\
8.12865497076023	-2.08628831361798e+20\\
8.1359649122807	-2.17705094446083e+20\\
8.14327485380117	-2.27176048124784e+20\\
8.15058479532164	-2.37058848987329e+20\\
8.1578947368421	-2.47371399104357e+20\\
8.16520467836257	-2.58132378408003e+20\\
8.17251461988304	-2.69361278478135e+20\\
8.17982456140351	-2.81078437795559e+20\\
8.18713450292398	-2.93305078525872e+20\\
8.19444444444444	-3.06063344900395e+20\\
8.20175438596491	-3.19376343263509e+20\\
8.20906432748538	-3.33268183858711e+20\\
8.21637426900585	-3.47764024428866e+20\\
8.22368421052632	-3.62890115709386e+20\\
8.23099415204678	-3.78673848896486e+20\\
8.23830409356725	-3.95143805176251e+20\\
8.24561403508772	-4.12329807403935e+20\\
8.25292397660819	-4.30262974026817e+20\\
8.26023391812866	-4.48975775347993e+20\\
8.26754385964912	-4.68502092232667e+20\\
8.27485380116959	-4.88877277362966e+20\\
8.28216374269006	-5.10138219151868e+20\\
8.28947368421053	-5.32323408431626e+20\\
8.29678362573099	-5.55473008037106e+20\\
8.30409356725146	-5.79628925409656e+20\\
8.31140350877193	-6.04834888352574e+20\\
8.3187134502924	-6.31136524074945e+20\\
8.32602339181287	-6.58581441666533e+20\\
8.33333333333333	-6.87219318152613e+20\\
8.3406432748538	-7.17101988284082e+20\\
8.34795321637427	-7.48283538224929e+20\\
8.35526315789474	-7.80820403306172e+20\\
8.3625730994152	-8.14771470022712e+20\\
8.36988304093567	-8.50198182457194e+20\\
8.37719298245614	-8.87164653322968e+20\\
8.38450292397661	-9.25737779826559e+20\\
8.39181286549708	-9.65987364558759e+20\\
8.39912280701754	-1.00798624163252e+21\\
8.40643274853801	-1.05181040829528e+21\\
8.41374269005848	-1.09753916225326e+21\\
8.42105263157895	-1.14525524495551e+21\\
8.42836257309941	-1.19504499109632e+21\\
8.43567251461988	-1.24699848460574e+21\\
8.44298245614035	-1.30120972140969e+21\\
8.45029239766082	-1.35777677925337e+21\\
8.45760233918129	-1.41680199489436e+21\\
8.46491228070176	-1.47839214898519e+21\\
8.47222222222222	-1.54265865897885e+21\\
8.47953216374269	-1.60971778040543e+21\\
8.48684210526316	-1.67969081688292e+21\\
8.49415204678363	-1.75270433924107e+21\\
8.50146198830409	-1.82889041415368e+21\\
8.50877192982456	-1.90838684269181e+21\\
8.51608187134503	-1.99133740922812e+21\\
8.5233918128655	-2.07789214114148e+21\\
8.53070175438597	-2.16820757979034e+21\\
8.53801169590643	-2.26244706324359e+21\\
8.5453216374269	-2.36078102127887e+21\\
8.55263157894737	-2.46338728318065e+21\\
8.55994152046784	-2.57045139889293e+21\\
8.5672514619883	-2.6821669741062e+21\\
8.57456140350877	-2.79873601988269e+21\\
8.58187134502924	-2.92036931745072e+21\\
8.58918128654971	-3.04728679882596e+21\\
8.59649122807017	-3.17971794394611e+21\\
8.60380116959064	-3.31790219503517e+21\\
8.61111111111111	-3.46208938894461e+21\\
8.61842105263158	-3.61254020825105e+21\\
8.62573099415205	-3.76952665192398e+21\\
8.63304093567251	-3.93333252641226e+21\\
8.64035087719298	-4.1042539580349e+21\\
8.64766081871345	-4.28259992760006e+21\\
8.65497076023392	-4.46869282821632e+21\\
8.66228070175439	-4.66286904730193e+21\\
8.66959064327485	-4.86547957384149e+21\\
8.67690058479532	-5.07689063198489e+21\\
8.68421052631579	-5.29748434213096e+21\\
8.69152046783626	-5.52765941068753e+21\\
8.69883040935673	-5.76783184975169e+21\\
8.70614035087719	-6.01843572800749e+21\\
8.71345029239766	-6.27992395419488e+21\\
8.72076023391813	-6.5527690945623e+21\\
8.7280701754386	-6.8374642257765e+21\\
8.73538011695906	-7.13452382482708e+21\\
8.74269005847953	-7.44448469752989e+21\\
8.75	-7.76790694730307e+21\\
8.75730994152047	-8.10537498596173e+21\\
8.76461988304094	-8.45749858835355e+21\\
8.7719298245614	-8.82491399273588e+21\\
8.77923976608187	-9.20828504887776e+21\\
8.78654970760234	-9.60830441595625e+21\\
8.79385964912281	-1.00256948124058e+22\\
8.80116959064327	-1.04612103199735e+22\\
8.80847953216374	-1.09156377443301e+22\\
8.81578947368421	-1.13897980346887e+22\\
8.82309941520468	-1.18845477649906e+22\\
8.83040935672515	-1.24007806793255e+22\\
8.83771929824561	-1.29394293043728e+22\\
8.84502923976608	-1.35014666317689e+22\\
8.85233918128655	-1.40879078734317e+22\\
8.85964912280702	-1.46998122930056e+22\\
8.86695906432749	-1.53382851167273e+22\\
8.87426900584795	-1.60044795271551e+22\\
8.88157894736842	-1.66995987433534e+22\\
8.88888888888889	-1.74248981912818e+22\\
8.89619883040936	-1.81816877682972e+22\\
8.90350877192983	-1.89713342058513e+22\\
8.91081871345029	-1.97952635346372e+22\\
8.91812865497076	-2.06549636566287e+22\\
8.92543859649123	-2.15519870286443e+22\\
8.9327485380117	-2.24879534622716e+22\\
8.94005847953216	-2.34645530451942e+22\\
8.94736842105263	-2.44835491891858e+22\\
8.9546783625731	-2.55467818102599e+22\\
8.96198830409357	-2.66561706467051e+22\\
8.96929824561403	-2.78137187209817e+22\\
8.9766081871345	-2.90215159517158e+22\\
8.98391812865497	-3.02817429222965e+22\\
8.99122807017544	-3.15966748128639e+22\\
8.99853801169591	-3.29686855027707e+22\\
9.00584795321637	-3.4400251850904e+22\\
9.01315789473684	-3.58939581615793e+22\\
9.02046783625731	-3.7452500844047e+22\\
9.02777777777778	-3.90786932740041e+22\\
9.03508771929824	-4.07754708658661e+22\\
9.04239766081871	-4.25458963649333e+22\\
9.04970760233918	-4.4393165368982e+22\\
9.05701754385965	-4.63206120892224e+22\\
9.06432748538012	-4.83317153609999e+22\\
9.07163742690059	-5.04301049150596e+22\\
9.07894736842105	-5.26195679206696e+22\\
9.08625730994152	-5.49040558123822e+22\\
9.09356725146199	-5.72876914127265e+22\\
9.10087719298246	-5.97747763636566e+22\\
9.10818713450292	-6.23697988801349e+22\\
9.11549707602339	-6.50774418398114e+22\\
9.12280701754386	-6.79025912233628e+22\\
9.13011695906433	-7.08503449206878e+22\\
9.1374269005848	-7.39260219188111e+22\\
9.14473684210526	-7.71351718880394e+22\\
9.15204678362573	-8.04835851836236e+22\\
9.1593567251462	-8.3977303280935e+22\\
9.16666666666667	-8.7622629662938e+22\\
9.17397660818713	-9.14261411795604e+22\\
9.1812865497076	-9.53946998994064e+22\\
9.18859649122807	-9.95354654751484e+22\\
9.19590643274854	-1.03855908044853e+23\\
9.20321637426901	-1.08363821692463e+23\\
9.21052631578947	-1.13067338491669e+23\\
9.21783625730994	-1.17974943158434e+23\\
9.22514619883041	-1.2309548833856e+23\\
9.23245614035088	-1.28438210557799e+23\\
9.23976608187134	-1.34012746863217e+23\\
9.24707602339181	-1.39829152185771e+23\\
9.25438596491228	-1.45897917455321e+23\\
9.26169590643275	-1.52229988500712e+23\\
9.26900584795322	-1.58836785768906e+23\\
9.27631578947368	-1.65730224898669e+23\\
9.28362573099415	-1.72922738185834e+23\\
9.29093567251462	-1.80427296978754e+23\\
9.29824561403509	-1.88257435044259e+23\\
9.30555555555556	-1.96427272946156e+23\\
9.31286549707602	-2.0495154348013e+23\\
9.32017543859649	-2.13845618210824e+23\\
9.32748538011696	-2.23125535158841e+23\\
9.33479532163743	-2.3280802768748e+23\\
9.3421052631579	-2.42910554641175e+23\\
9.34941520467836	-2.53451331789883e+23\\
9.35672514619883	-2.64449364635964e+23\\
9.3640350877193	-2.75924482642591e+23\\
9.37134502923977	-2.87897374945273e+23\\
9.37865497076023	-3.00389627610724e+23\\
9.3859649122807	-3.13423762510112e+23\\
9.39327485380117	-3.27023277876616e+23\\
9.40058479532164	-3.41212690620254e+23\\
9.4078947368421	-3.56017580476097e+23\\
9.41520467836257	-3.71464636065274e+23\\
9.42251461988304	-3.87581702951645e+23\\
9.42982456140351	-4.04397833780555e+23\\
9.43713450292398	-4.21943340589877e+23\\
9.44444444444444	-4.40249849387417e+23\\
9.45175438596491	-4.59350357092857e+23\\
9.45906432748538	-4.79279290946638e+23\\
9.46637426900585	-5.00072570492649e+23\\
9.47368421052632	-5.21767672246182e+23\\
9.48099415204678	-5.44403697163466e+23\\
9.48830409356725	-5.68021441034119e+23\\
9.49561403508772	-5.92663467923101e+23\\
9.50292397660819	-6.18374186794259e+23\\
9.51023391812866	-6.45199931453243e+23\\
9.51754385964912	-6.73189043953562e+23\\
9.52485380116959	-7.02391961615752e+23\\
9.53216374269006	-7.32861307816155e+23\\
9.53947368421053	-7.64651986708541e+23\\
9.54678362573099	-7.97821282048901e+23\\
9.55409356725146	-8.32428960301115e+23\\
9.56140350877193	-8.68537378208879e+23\\
9.5687134502924	-9.06211595027297e+23\\
9.57602339181287	-9.45519489615947e+23\\
9.58333333333333	-9.8653188260393e+23\\
9.5906432748538	-1.02932266384656e+24\\
9.59795321637427	-1.07396892540281e+24\\
9.60526315789474	-1.12055110027267e+24\\
9.6125730994152	-1.16915310714367e+24\\
9.61988304093567	-1.21986250140703e+24\\
9.62719298245614	-1.27277063271467e+24\\
9.63450292397661	-1.3279728093605e+24\\
9.64181286549708	-1.38556846978138e+24\\
9.64912280701754	-1.44566136148614e+24\\
9.65643274853801	-1.50835972773416e+24\\
9.66374269005848	-1.57377650229924e+24\\
9.67105263157895	-1.64202951266866e+24\\
9.67836257309941	-1.71324169204289e+24\\
9.68567251461988	-1.78754130051676e+24\\
9.69298245614035	-1.86506215583987e+24\\
9.70029239766082	-1.9459438741708e+24\\
9.70760233918129	-2.03033212125797e+24\\
9.71491228070176	-2.11837887449858e+24\\
9.72222222222222	-2.21024269634644e+24\\
9.72953216374269	-2.30608901956036e+24\\
9.73684210526316	-2.40609044480547e+24\\
9.74415204678363	-2.51042705114241e+24\\
9.75146198830409	-2.61928671996241e+24\\
9.75877192982456	-2.73286547295029e+24\\
9.76608187134503	-2.85136782468282e+24\\
9.7733918128655	-2.97500715049591e+24\\
9.78070175438597	-3.10400607028187e+24\\
9.78801169590643	-3.23859684890614e+24\\
9.7953216374269	-3.37902181396325e+24\\
9.80263157894737	-3.52553379162247e+24\\
9.80994152046784	-3.67839656134636e+24\\
9.8172514619883	-3.83788533029916e+24\\
9.82456140350877	-4.00428722829764e+24\\
9.83187134502924	-4.17790182419332e+24\\
9.83918128654971	-4.35904166461418e+24\\
9.84649122807017	-4.54803283603357e+24\\
9.85380116959064	-4.74521555117621e+24\\
9.86111111111111	-4.95094476081472e+24\\
9.86842105263158	-5.16559079205595e+24\\
9.87573099415205	-5.38954001426359e+24\\
9.88304093567251	-5.62319553381345e+24\\
9.89035087719298	-5.86697791892954e+24\\
9.89766081871345	-6.12132595590289e+24\\
9.90497076023392	-6.3866974380517e+24\\
9.91228070175439	-6.6635699888399e+24\\
9.91959064327485	-6.95244192063279e+24\\
9.92690058479532	-7.25383313063219e+24\\
9.93421052631579	-7.56828603560049e+24\\
9.94152046783626	-7.89636654705243e+24\\
9.94883040935673	-8.23866508866619e+24\\
9.95614035087719	-8.59579765774138e+24\\
9.96345029239766	-8.96840693260997e+24\\
9.97076023391813	-9.35716342798958e+24\\
9.9780701754386	-9.76276670035372e+24\\
9.98538011695906	-1.01859466054842e+25\\
9.99269005847953	-1.06274646104635e+25\\
10	-1.10881151624648e+25\\
};
\addlegendentry{y1}

\end{axis}
\end{tikzpicture}%
    \end{tikzpicture}}
    \caption{Reponse of the first output of the interconnection between $\mathbf{N}$ and $\mathbf{\Delta}$ to step input.}
    \label{Fres}
\end{figure}

The interconnection between the closed-loop and $\mathbf{\Delta}$ is given by $$\bm{F} = \mathcal{F}_u(\bm{N},\bm{\Delta})$$ which is unstable, as seen in Figure \ref{Fres}.
